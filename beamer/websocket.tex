\documentclass{beamer}
\usepackage[utf8x]{inputenc}
\usepackage[ngerman]{babel}
\usepackage{amsmath}
\usepackage{amsfonts}
\usepackage{amssymb}
\usepackage{graphicx}
\usepackage{subfigure}
\author{Johannes Hackel\and Falco Prescher}
\title{WebSocket}

\usetheme{Ilmenau}
\useoutertheme{infolines}
\usecolortheme{rose}

\begin{document}

\begin{frame}
\titlepage
\end{frame}

\begin{frame}
\frametitle{Gliederung}
\tableofcontents
\end{frame}

\section{Definition}
\begin{frame}
\frametitle{Definition}
\end{frame}

\section{Warum?}
\begin{frame}
\frametitle{Warum?}

\begin{itemize}
\item immer mehr Anwendungen durch Webtechnologien
\item fehlende Echtzeit
\item vorher durch Polling/Long Polling
\item führt zu viel Overhead und Netzwerkauslastung
\end{itemize}

\end{frame}


\begin{frame}
\frametitle{Polling/Long Polling}
\begin{figure}
\subfigure{\includegraphics[width=5cm]{polling.png}}
\subfigure{\includegraphics[width=5cm]{long_polling.png}}
\end{figure}
\end{frame}

\begin{frame}
\begin{figure}
\begin{center}
\includegraphics[width=12cm]{poll-ws-compare.png}
\end{center}
\end{figure}
\end{frame}

\section{Verwendungen}
\begin{frame}
\frametitle{Verwendungen}
\begin{itemize}
\item Facebook, Twitter
\item Chats
\item kollaborative Websites (etherpad)
\item Spiele auf HTML5-Basis
\end{itemize}
\end{frame}

\section{Bestandteile}

\subsection{Netzwerkprotokoll}
\begin{frame}
\begin{figure}[htbp]
\frametitle{Bestandteile}
\framesubtitle{Netzwerkprotokoll}
\begin{minipage}[t]{5cm}
\vspace{0pt}
\begin{itemize}
\item Standard-HTTP-Ports 80 und 443
\item auch durch Firewalls hinweg
\item Verbindungsaufbau durch WebSocket Protocol Handshake
\end{itemize}
\end{minipage}
\hfill
\begin{minipage}[t]{6cm}
\vspace{0pt}
\includegraphics[width=6cm]{WebSocket.png}
\end{minipage}
\end{figure}
\end{frame}

\begin{frame}
\frametitle{Verbindungsaufbau}
\framesubtitle{Client}
GET /chatService HTTP/1.1\\
Host: server.example.com\\
Upgrade: websocket\\
Connection: Upgrade\\
Sec-WebSocket-Key: dGhlIHNhbXBsZSBub25jZQ==\\
Sec-WebSocket-Origin: http://example.com\\
Sec-WebSocket-Protocol: chat, superchat\\
Sec-WebSocket-Version: 8 \\
\end{frame}

\begin{frame}
\frametitle{Verbindungsaufbau}
\framesubtitle{Server}
HTTP/1.1 101 Switching Protocols\\
Upgrade: websocket\\
Connection: Upgrade\\
Sec-WebSocket-Accept: s3pPLMBiTxaQ9kYGzzhZRbK+xOo=\\
Sec-WebSocket-Protocol: superchat\\
\end{frame}

\begin{frame}
\frametitle{WebSocket Protocol}
\begin{itemize}
\item geringer Overhead (zwei Byte)
\item Begin: 0x00
\item Ende: OxFF
\item Daten in UTF-8
\item Beispiel: \textbackslash 0x00Hello, WebSocket\textbackslash 0xff
\end{itemize}
\end{frame}

\subsection{Javascript Client-API}
\begin{frame}
\frametitle{Bestandteile}
\framesubtitle{Javascript Client-API}
\end{frame}

\section{Implementierungen/Technologien}
\begin{frame}
\frametitle{Implementierungen/Technologien}
\framesubtitle{Server}
\begin{itemize}
\item Apache httpd (via "mod\_pywebsocket")
\item Jetty
\item jWebSocket
\item Kaazing WebSocket Gateway
\item Oracle Glassfish 3.1
\item Netty Project
\item Node.js/Socket.io
\end{itemize}
\end{frame}

\section{Zusammenfassung}
\begin{frame}
\frametitle{Zusammenfassung}
\end{frame}

\section{Quellen}
\begin{frame}
\frametitle{Quellen}
\begin{itemize}
\item http://www.heise.de/developer/artikel/WebSocket-Annaeherung-an-Echtzeit-im-Web-1260189.html
\item http://www.websocket.org
\end{itemize}
\end{frame}

\end{document}